% Created 2022-04-05 Tue 11:55
% Intended LaTeX compiler: pdflatex
\documentclass[presentation]{beamer}
\usepackage[utf8]{inputenc}
\usepackage[T1]{fontenc}
\usepackage{graphicx}
\usepackage{grffile}
\usepackage{longtable}
\usepackage{wrapfig}
\usepackage{rotating}
\usepackage[normalem]{ulem}
\usepackage{amsmath}
\usepackage{textcomp}
\usepackage{amssymb}
\usepackage{capt-of}
\usepackage{hyperref}
\usetheme{Antibes}
\author{Aditya Athalye}
\date{2022-03-26 Sat}
\title{n ways to FizzBuzz in Clojure}
\subtitle{(Or, How to Stop Worrying and Start Decomplecting.)}
\hypersetup{
 pdfauthor={Aditya Athalye},
 pdftitle={n ways to FizzBuzz in Clojure},
 pdfkeywords={},
 pdfsubject={},
 pdfcreator={Emacs 26.3 (Org mode 9.3.7)},
 pdflang={English}}
\begin{document}

\maketitle
\section{Intro}
\label{sec:org62bc719}
\begin{frame}[label={sec:org6be44f6}]{Demo Clojure concepts \& stdlib via FizzBuzz}
\begin{itemize}
\item \alert{The Demo Plan}: pray to Demogods and\ldots{}
\begin{itemize}
\item Do rapid-fire live demo, until timer runs out
\item Where each FizzBuzz has reason to exist
\item And \emph{"decomplect"} is said so often you think it's normal
\end{itemize}
\end{itemize}
\pause
\begin{itemize}
\item \alert{The Demofail Plan}: Blog post (may also upload video)
\end{itemize}
\pause
\begin{itemize}
\item \alert{The Actual Plan}: Drop some sizzlin' hot takes. Leik dis\ldots{}
\end{itemize}

\begin{verse}
If you do FP right, you get OOP for free and vice-versa.\\
\#Smalltalk \#Clojure \#Erlang \#OCaml \#Haskell\\
- Yours Truly :)\\
\end{verse}
\end{frame}
\section{Prayer}
\label{sec:org80416a7}
\begin{frame}[label={sec:orge09d711}]{O Lambda the Ultimate, bless we who are in this demo\ldots{}}
\begin{verse}
That our core be functional,\\
\hspace*{2em}and our functions be pure.\\
That our data be immutable,\\
\hspace*{2em}so we may know the value of values.\\
That our systems be composable,\\
\hspace*{2em}so they may scale with grace.\\
That their States only mutate\\
\hspace*{2em}in pleasantly surprising ways.\\
That the networks and servers stay up.\\
\hspace*{2em}Well, at least through this demo.\\
\vspace*{1em}
For otherwise, nothing lives, nothing evolves.\\
\vspace*{1em}
In the name of the \(\alpha\) and the \(\beta\) and the \(\eta\)\ldots{}\\
\hspace*{2em}(\(\lambda\)x.x x) (\(\lambda\)x.x x) ; eternally\\
\end{verse}
\end{frame}
\section{Definitions}
\label{sec:org1cbf7d2}
\begin{frame}[label={sec:orgaba8d91},fragile]{Definitions}
 \begin{itemize}
\item \alert{FizzBuzz}
\end{itemize}
\begin{quote}
Fizz buzz is a group word game for children
to teach them about division. Players take
turns to count incrementally, replacing any
number divisible by \texttt{THREE} with the word
"Fizz", and any number divisible by \texttt{FIVE}
with the word "Buzz". \emph{- Wikipedia}
\end{quote}
\pause
\begin{itemize}
\item \alert{"Numbers"}: Natural Numbers starting at 1
\end{itemize}
\pause
\begin{itemize}
\item \alert{\emph{Clojurish}}: postmodern revival of
Latin roots of US English
\begin{itemize}
\item \emph{complect}      : braided, entwined, hopelessly
\item \emph{complected}    : Thing that makes Clojurian frown.
\item \emph{decomplect}    : unbraid, disentwine
\item \emph{decomplected}  : Thing that makes Clojurian smile.
\item \emph{decomplecting} : Activity that Clojurian enjoys.
(coming up!)
\end{itemize}
\end{itemize}
\end{frame}
\section{A Classic}
\label{sec:org1f2beca}
\begin{frame}[label={sec:orgaeb2465},fragile]{Le FizzBuzz Classique: First try}
 \begin{itemize}
\item Accidentally discover \emph{\texttt{for}} in stdlib
\item "Ooh, List Comprehension. Nice!"
(- The Python gentlenerds in the house :)
\pause
\begin{verbatim}
(ns user)
(defn fizz-buzz-classic
  [num-xs]
  (for [n num-xs]
    (cond
      (zero? (rem n 15)) (println "FizzBuzz")
      (zero? (rem n 3)) (println "Fizz")
      (zero? (rem n 5)) (println "Buzz")
      :else (println n))))
\end{verbatim}
\pause
\item Those pesky nils, tho.. (see REPL console)
\end{itemize}
\end{frame}
\begin{frame}[label={sec:org401141f},fragile]{Le FizzBuzz Classique est mort à Clojure}
 Désolé :(

\begin{itemize}
\item \emph{\texttt{for}} is Lazy

\item REPL is eager

\item Here be Dragons

\item Unlearn old habits

\item Or learn from prod outage
\end{itemize}
\end{frame}
\begin{frame}[label={sec:orgd4a6253},fragile]{Le FizzBuzz Classique: Remedied}
 No more \emph{\texttt{println}}, no more nils.
\begin{verbatim}
(defn lazybuzz
  [num-xs]
  (for [n num-xs]
    (cond
      (zero? (rem n 15)) "FizzBuzz"
      (zero? (rem n 3)) "Fizz"
      (zero? (rem n 5)) "Buzz"
      :else n)))

(lazybuzz [1 3 5 15 16]) ; yes

(fizz-buzz-classic [1 3 5 15 19]) ; bleh
\end{verbatim}
\end{frame}
\begin{frame}[label={sec:orgb707d02},fragile]{Le FizzBuzz Classique: dissected}
 \begin{itemize}
\item "Classic" FizzBuzz considered harmful (in Clojure)

\item Examine \& avoid its severe defects:
\begin{itemize}
\item \alert{Broken behaviour}
\begin{itemize}
\item calculations functional
\item \texttt{println} non-deterministic
\end{itemize}

\item \alert{Broken API contract}
\begin{itemize}
\item "Classic" version returns useless nils
\item \texttt{lazybuzz} returns useful values
\item We like useful values
\end{itemize}

\item \alert{Broken time model}
\begin{itemize}
\item Effects ("do NOW") + Laziness ("maybe never") = Bad!
\item Define separately, join later in safe ways
\end{itemize}

\item \alert{Broken aesthetic}
\begin{itemize}
\item Do one job, do it well. Printing is \emph{second} job.
\item "That's George's problem." - Hal \& Gerry
\end{itemize}
\end{itemize}

\item Bonus: See blog post for ideas to get \emph{fired} with fizzbuzz.
\end{itemize}
\end{frame}
\begin{frame}[label={sec:orgf26ca7d},fragile]{Le FizzBuzz Classique: resurrected, the Clojure way}
 \begin{itemize}
\item Keep your fns pure, like \texttt{lazybuzz}

\item Laziness becomes friend, as nice bonus!
(Recall the children's game definition)
\begin{verbatim}
(def all-naturals (rest (range)))
(def all-fizz-buzzes (lazybuzz all-naturals))
\end{verbatim}

\item Let REPL print. Separately.
\begin{verbatim}
(take 15 all-fizz-buzzes)
\end{verbatim}
\end{itemize}
\end{frame}
\section{decomplect choice}
\label{sec:orgb43e970}
\begin{frame}[label={sec:orgd9693b6},fragile]{decomplect sequence-making v/s choice-making}
 \begin{itemize}
\item Lift out logic as its own definition
\begin{itemize}
\item "Do one thing well"
\begin{verbatim}
(defn basic-buzz [n]
  (cond
    (divisible? n 15) "FizzBuzz"
    (divisible? n 3) "Fizz"
    (divisible? n 5) "Buzz"
    :else n))
\end{verbatim}
\item Retain composability with \texttt{for}
\begin{verbatim}
(def all-fizz-buzzes
  (for [n (rest (range))]
    (basic-buzz n)))
\end{verbatim}
\item \emph{And} open up design space
\begin{verbatim}
(def all-fizz-buzzes
  (map basic-buzz (rest (range))))
\end{verbatim}
\item \texttt{reduce} is homework (lazy v/s eager)
\end{itemize}
\end{itemize}
\end{frame}
\section{decomplect CPU}
\label{sec:org060b8b9}
\begin{frame}[label={sec:orgef7b39f},fragile]{decomplect execution (CPU-bound parallelism)}
 \begin{itemize}
\item Almost too embarrassing to write\ldots{}
\begin{verbatim}
(def fizz-buzz map)
(def par-buzz pmap)
\end{verbatim}
\item Get CPU-bound parallelism trivially\ldots{}
\begin{verbatim}
(= (fizz-buzz basic-buzz (range 1 101))
   (par-buzz basic-buzz (range 1 101)))
\end{verbatim}
\item Not too hard to understand!
\begin{verbatim}
(clojure.repl/source pmap)
\end{verbatim}
\end{itemize}
\end{frame}
\section{decomplect domain: solution side as well as problem side}
\label{sec:orgac55a0f}
\begin{frame}[label={sec:org4be9531}]{decomplect domain: solution side as well as problem side}
\begin{itemize}
\item "Solution side" => the language of the domain
\begin{itemize}
\item Function names
\item APIs and contracts
\item Domain abstractions and entity relationships
\end{itemize}
\item "Problem side" => the nature of the domain
\begin{itemize}
\item Direct ("declarative") expression of middle-school maths
\item Pry apart the "what" from the "how"
\end{itemize}
\end{itemize}
\end{frame}
\begin{frame}[label={sec:org4627a71},fragile]{decomplect solution domain (concept of divisibility)}
 \begin{itemize}
\item Name locally or lift to top level?
\begin{itemize}
\item We can let-bind a lambda
\begin{verbatim}
(defn letbuzz [num-xs]
  (for [n num-xs]
    (let [divisible? (fn [n1 n2] (zero? (rem n1 n2)))]
      (cond
        (divisible? n 15) "FizzBuzz"
        (divisible? n 3) "Fizz"
        (divisible? n 5) "Buzz"
        :else n))))
\end{verbatim}
\item But we like tiny fns that add compositional firepower
\begin{verbatim}
(defn divisible? [n1 n2]
  (zero? (rem n1 n2)))

(def divisible? (comp zero? rem))
\end{verbatim}
\end{itemize}
\item All 3 variants are refrentially transparent
\end{itemize}
\end{frame}
\begin{frame}[label={sec:orga37e5c2},fragile]{decomplect solution domain more (language of fizzbuzz)}
 \begin{itemize}
\item Open up design space more with more domain concepts
\begin{verbatim}
(defn divisible?
  "Return the-word (truthy) when n divisible,
  nil otherwise (falsey)."
  [divisor the-word n]
  (when (zero? (rem n divisor))
    the-word))

(def fizzes? (partial divisible? 3 "Fizz"))
(def buzzes? (partial divisible? 5 "Buzz"))
(def fizzbuzzes? (partial divisible? 15 "FizzBuzz"))
\end{verbatim}
\item Note:
\begin{itemize}
\item Truthiness/falseyness
\item args list ordered as more constant -to-> more variable
\end{itemize}
\end{itemize}
\end{frame}
\begin{frame}[label={sec:orgc8cad74},fragile]{decomplect solution domain more (language of fizzbuzz)}
 \begin{itemize}
\item Now we can do \emph{\texttt{or}} buzz
\begin{itemize}
\item \emph{\texttt{or}} is short-circuiting
\begin{verbatim}
(defn or-buzz [n]
  (or (fizzbuzzes? n)
      (buzzes? n)
      (fizzes? n)
      n))
\end{verbatim}
\end{itemize}
\item Or, \emph{\texttt{juxt}} express our choice
\begin{itemize}
\item given \emph{\texttt{((juxt f g h) 42)} -> \texttt{[(f 42) (g 42) (h 42)]}}
\begin{verbatim}
(defn juxt-buzz [n]
  (some identity
        ((juxt fizzbuzzes? buzzes? fizzes? identity)
         n)))
\end{verbatim}
\item Here \texttt{juxt} is too subtle for production, BUT useful later
\end{itemize}
\item Sadly, order of conditionals still matters in both cases
\end{itemize}
\end{frame}
\begin{frame}[label={sec:org9629f8b},fragile]{decomplect problem domain (school maths, 15 is LCM)}
 \begin{itemize}
\item Make order of calculation \emph{not} matter
\item A table of remainders of 15, in a hash-map
\begin{verbatim}
(def rem15->fizz-buzz
  {0  "FizzBuzz"
   3  "Fizz"
   6  "Fizz"
   9  "Fizz"
   12 "Fizz"
   5  "Buzz"
   10 "Buzz"})
\end{verbatim}
\item Maps are functions too!
\begin{verbatim}
(rem15->fizz-buzz (rem 3 15))
;; ~nil~ implies "no result found"
(rem15->fizz-buzz (rem 1 15))
\end{verbatim}
\end{itemize}
\end{frame}
\begin{frame}[label={sec:org5aae148},fragile]{decomplect problem domain (school maths, 15 is LCM)}
 \begin{itemize}
\item \texttt{nil}-pun with short-circuiting \alert{\texttt{or}}
\begin{verbatim}
(defn or-rem15-buzz
  [n]
  (or (rem15->fizz-buzz (rem n 15))
      n))
\end{verbatim}

\item But \alert{\texttt{get}} is more right, with fallback for "not found"
\begin{verbatim}
(defn get-rem15-buzz
  [n]
  (get rem15->fizz-buzz
       (rem n 15)
       n))
\end{verbatim}

\item And we can do
\begin{verbatim}
(fizz-buzz get-rem15-n (range 1 16))
\end{verbatim}
\end{itemize}
\end{frame}
\begin{frame}[label={sec:orge65911d},fragile]{decomplect problem domain more (modulo math)}
 \begin{itemize}
\item FizzBuzz cyclical in modulo 3, 5, 15
\begin{verbatim}
(def mod-cycle-buzz ; sequence ordered by definition
  (let [n  identity
        f  (constantly "Fizz")
        b  (constantly "Buzz")
        fb (constantly "FizzBuzz")]
    (cycle [n n f n b f n n f b n f n n fb])))
\end{verbatim}
\item Map can operate over \texttt{n} collections.
\begin{verbatim}
(def all-fizz-buzzes
  (map (fn [f n] (f n))
       mod-cycle-buzz ; countless modulo pattern
       (rest (range)))) ; countless natural numbers
\end{verbatim}
\end{itemize}
\end{frame}
\begin{frame}[label={sec:orge3d262c},fragile]{decomplect ALL the FizzBuzzes (prime factors)}
 \begin{itemize}
\item Think prime factors and modulo cycles
\begin{itemize}
\item e.g. \texttt{[nil nil "Fizz"]}, \texttt{[nil nil nil nil "Buzz"]}
\begin{verbatim}
(defn any-mod-cycle-buzz
  [num & words]
  (or (not-empty (reduce str words))
      num))
\end{verbatim}
\item Recall \emph{\texttt{map}} is variadic, so bring on all the primes!
\begin{verbatim}
(map any-mod-cycle-buzz
     (range 1 16)
     (cycle [nil nil "Fizz"])
     (cycle [nil nil nil nil "Buzz"])
     (cycle [nil "Biz"])
     (cycle [nil nil nil nil nil nil "Fuz"]))
\end{verbatim}
\item Bonus: get \emph{\texttt{identity}} (\texttt{I}) definition too:
\texttt{I} of \texttt{+} is 0, \texttt{I} of \texttt{*} is 1, \texttt{I} of \texttt{FizzBuzz} is all naturals
\begin{verbatim}
(map any-mod-cycle-buzz (range 1 16))
\end{verbatim}
\end{itemize}
\end{itemize}
\end{frame}
\section{decomplect \emph{mechanism} and \emph{policy}}
\label{sec:org81aebbb}
\begin{frame}[label={sec:org2eda929},fragile]{decomplect \emph{mechanism} and \emph{policy} (Say what?)}
 \begin{itemize}
\item Classically, "mechanism" and "policy" hard-wired together
\begin{verbatim}
<-- ------- MECHANISM -------- -->|<-- POLICY -->

| n divisible? 3 | n divisible? 5 | Final value |
|----------------+----------------+-------------|
| true           | true           | FizzBuzz    |
| true           | false          | Fizz        |
| false          | true           | Buzz        |
| false          | false          | n           |
\end{verbatim}
\end{itemize}
\end{frame}
\begin{frame}[label={sec:org91ec3d5},fragile]{decomplect \emph{mechanism} and \emph{policy} (pry the two apart)}
 \begin{itemize}
\item \alert{Mechanism}: the way to construct \emph{a} truth table
\begin{verbatim}
(ns dispatch.buzz)
(defn mechanism
  "Given two fns, presumably of any-to->Boolean,
   return a fn that can construct inputs of a
   2-input truth table."
  [f? g?]
  (juxt f? g?))
\end{verbatim}
\item \alert{Policy}: the way to calculate Fizz Buzz
\begin{verbatim}
(defn divisible? [divisor n]
  (zero? (rem n divisor)))

(def fizzes? (partial divisible? 3))
(def buzzes? (partial divisible? 5))
\end{verbatim}
\end{itemize}
\end{frame}
\begin{frame}[label={sec:org46a9058},fragile]{decomplect \emph{mechanism} and \emph{policy} (recompose a-la-carte)}
 \begin{itemize}
\item \alert{Mechanism + Policy}: Polymorphic dispatch joins
truth table mechanism with FizzBuzz policy
\begin{itemize}
\item Key: specialise the truth table mechanism to FizzBuzz
\begin{verbatim}
(map (mechanism fizzes? buzzes?)
     [15 3 5 1])
\end{verbatim}
\end{itemize}
\item Use in dispatch mechanism
\begin{itemize}
\item connect truth table rows to results
\begin{verbatim}
(def fizz-buzz map)
(def fizz-buzz-mecha (mechanism fizzes? buzzes?))
(defmulti dispatch-buzz
  "Each method yields result record
  for truth table record."
  fizz-buzz-mecha)
\end{verbatim}
\end{itemize}
\end{itemize}
\end{frame}
\begin{frame}[label={sec:orge4cc296},fragile]{decomplect \emph{mechanism} and \emph{policy} (recompose a-la-carte)}
 \begin{itemize}
\item \alert{Mechanism + Policy}: Yes, 'tis a wee FizzBuzz interpreter!
\begin{verbatim}
(defmethod dispatch-buzz [true true]
  [n]
  "FizzBuzz")
(defmethod dispatch-buzz [true false]
  [n]
  "Fizz")
(defmethod dispatch-buzz [false true]
  [n]
  "Buzz")
(defmethod dispatch-buzz :default
  [n]
  n)
(fizz-buzz dispatch-buzz [1 3 5 15 16])
\end{verbatim}
\end{itemize}
\end{frame}
\section{decomplect OOP}
\label{sec:orge77f11f}
\begin{frame}[label={sec:org18fbbad}]{decomplect OOP: What is complected?}
Classical OOP complects these things:
\begin{itemize}
\item Name (Class name / Java type)
\item Structure (Class members, methods etc.)
\item Behaviour (effects caused by methods)
\item State (contained in the run-time instance of the Class)
\end{itemize}
\end{frame}
\begin{frame}[label={sec:orgdcce424},fragile]{decomplect OOP: with Clojure Polymorphism}
 \begin{itemize}
\item Bring back usual suspects
\begin{verbatim}
(ns oops.fizzbuzz)
(def divisible? (comp zero? rem))
(def fizz-buzz map)
(defn basic-buzz [n]
  (cond
    (divisible? n 15) "FizzBuzz"
    (divisible? n 3) "Fizz"
    (divisible? n 5) "Buzz"
    :else n))
\end{verbatim}
\item Introduce protocols (like Java Interfaces, but better)
\begin{verbatim}
(defprotocol IFizzBuzz
  (proto-buzz [this]))
\end{verbatim}
\end{itemize}
\end{frame}
\begin{frame}[label={sec:orgf1eb234},fragile]{decomplect OOP: with Clojure Polymorphism}
 \begin{itemize}
\item Add new behaviour to existing types including \emph{any} Java builtin
\begin{verbatim}
(extend-protocol IFizzBuzz
  java.lang.Number
  (proto-buzz [this]
    (basic-buzz this)))
\end{verbatim}
\item Like this: Java type-based Polymorphic dispatch
\begin{verbatim}
(fizz-buzz proto-buzz [1 3 5 15 16])
(fizz-buzz proto-buzz [1.0 3.0 5.0 15.0 15.9])
\end{verbatim}
\end{itemize}
\end{frame}
\begin{frame}[label={sec:orga0b0dec},fragile]{decomplect OOP: with Clojure Polymorphism}
 \begin{itemize}
\item Clojure protocols cleanly solve the \uline{\emph{\href{https://en.wikipedia.org/wiki/Expression\_problem}{Expression Problem}}}
\item \emph{\alert{Without}} breaking Equality or any other existing semantics
\begin{verbatim}
(= 42 42) ; => true (long and long)
(= 42 42.0) ; => false (long and double)
(= 42.0 42.0) ; => true (double and double)

;; False, as it should be.
(= (fizz-buzz proto-buzz [1 3 5 15 16])
   (fizz-buzz proto-buzz [1.0 3.0 5.0 15.0 15.9]))
\end{verbatim}
\item Without performance overhead (JVM hotspot optimization)
\end{itemize}
\end{frame}
\section{decomplect information (nondestructive fizzbuzz)}
\label{sec:org3bdf635}
\begin{frame}[label={sec:org0426aee}]{decomplect information (nondestructive fizzbuzz)}
\begin{itemize}
\item All fizz-buzzes so far \emph{lose information}
\item Can't undo entropy
\item Very Very Bad (especially in an age of plentiful memory)
\item We can FizzBuzz with "Composite" Data
\end{itemize}
\end{frame}
\begin{frame}[label={sec:org64d507b},fragile]{decomplect information (Peano arithmetic representation)}
 \begin{itemize}
\item Define PeanoBuzz number representation starting at \texttt{[0 0]}
\item PeanoBuzz is closed under this definition of Successor (\emph{\texttt{S}})
\begin{verbatim}
(def S (comp (juxt identity get-rem15-buzz)
             inc
             first))
(def all-peano-buzzes (iterate S [0 0]))
\end{verbatim}
\item This is nondestructive (we don't lose our Numbers)
\begin{verbatim}
(take 16 all-peano-buzzes)
\end{verbatim}
\item Trivially map PeanoBuzz back to Standard FizzBuzz
\begin{verbatim}
(= (fizz-buzz basic-buzz (range 1 101))
   (fizz-buzz second
              (take 100 (rest all-peano-buzzes))))
\end{verbatim}
\end{itemize}
\end{frame}
\begin{frame}[label={sec:org02c3153},fragile]{decomplect information (Records to represent FizzBuzz)}
 \begin{itemize}
\item Records provide Java Types + all generic hash-map properties
\begin{verbatim}
(ns boxed.fizz.buzz)
(defrecord Fizz [n])
(defrecord Buzz [n])
(defrecord FizzBuzz [n])
(defrecord Identity [n])
\end{verbatim}
\end{itemize}
\end{frame}
\begin{frame}[label={sec:orgd5c1533},fragile]{decomplect information (Records to represent FizzBuzz)}
 \begin{itemize}
\item Boxed variant of \texttt{basic-buzz}
\begin{verbatim}
(def divisible? (comp zero? rem))
(def fizz-buzz map)

(defn boxed-buzz [n]
  (cond
    (divisible? n 15) (->FizzBuzz n)
    (divisible? n 3) (->Fizz n)
    (divisible? n 5) (->Buzz n)
    :else (->Identity n)))

(def all-boxed-buzzes
  (map boxed-buzz (rest (range))))
\end{verbatim}
\end{itemize}
\end{frame}
\begin{frame}[label={sec:org0ea2c92},fragile]{decomplect information (Records to represent FizzBuzz)}
 \begin{itemize}
\item Composite hash-map-like data!
\begin{verbatim}
(= (fizz-buzz boxed-buzz [1 3 5 15])
   [#boxed.fizz.buzz.Identity{:n 1}
    #boxed.fizz.buzz.Fizz{:n 3}
    #boxed.fizz.buzz.Buzz{:n 5}
    #boxed.fizz.buzz.FizzBuzz{:n 15}])
\end{verbatim}
\item Which is nondestructive!!
\begin{verbatim}
(= [1 3 5 15]
   (fizz-buzz (comp :n boxed-buzz) [1 3 5 15]))
\end{verbatim}
\item \emph{And} which has real Java types!!!
\begin{verbatim}
(= (map type (fizz-buzz boxed-buzz [1 3 5 15]))
   [boxed.fizz.buzz.Identity
    boxed.fizz.buzz.Fizz
    boxed.fizz.buzz.Buzz
    boxed.fizz.buzz.FizzBuzz])
\end{verbatim}
\end{itemize}
\end{frame}
\section{decomplect context (whence a number FizzBuzzes)}
\label{sec:orgac0c86a}
\begin{frame}[label={sec:org0d6376c}]{decomplect context (whence a number FizzBuzzes)}
\begin{itemize}
\item Context thus far was run-time calculation
\begin{itemize}
\item Meaning embedded in in-line logic
\item Not optional / situational by default
\end{itemize}
\item Some ideas to pull out FizzBuzz interpretation \emph{context}
\begin{itemize}
\item Super handy in \emph{some} situations
\item Utility is is contextual
\end{itemize}
\end{itemize}
\end{frame}
\begin{frame}[label={sec:org73848a5},fragile]{decomplect context (parse, don't calculate or interpret)}
 \begin{itemize}
\item Off-label use of \_\href{https://clojure.org/guides/spec}{Clojure Spec\_}'s \emph{\texttt{conform}} as parser
\item Skirts the "can be a very bad idea" territory. YMMV.
\begin{verbatim}
(ns conformer.buzz)
(require '[clojure.spec.alpha :as s])

(defn divisible? [divisor n]
  (zero? (rem n divisor)))
(def fizzes? (partial divisible? 3))
(def buzzes? (partial divisible? 5))

(s/def ::number number?)
(s/def ::fizzes (s/and ::number fizzes?))
(s/def ::buzzes (s/and ::number buzzes?))
\end{verbatim}
\end{itemize}
\end{frame}
\begin{frame}[label={sec:orgbec344a},fragile]{decomplect context (parse, don't calculate or interpret)}
 \begin{itemize}
\item Now we can parse input data\ldots{}
\begin{verbatim}
(s/conform ::fizzes 3) ; 3
(s/conform ::buzzes 5) ; 5
(s/conform ::buzzes 3) ; :clojure.spec.alpha/invalid
(s/conform (s/and ::fizzes ::buzzes) 15) ; 15
\end{verbatim}
\item \emph{And} handle non-conforming data gracefully,
instead of panicking and throwing exceptions
\begin{verbatim}
(s/conform (s/or ::fizzes ::buzzes) "lol")
;; => :clojure.spec.alpha/invalid
\end{verbatim}
\end{itemize}
\end{frame}
\begin{frame}[label={sec:orgda6109e},fragile]{decomplect context (parse, don't calculate or interpret)}
 \begin{itemize}
\item Relate numbers, parsers, parser results
\item Set of FizzBuzz parsers
\begin{verbatim}
(def fizz-buzz-specs #{::fizzes ::buzzes ::number})
\end{verbatim}
\item Parser-accumulator
\begin{verbatim}
(defn spec-parse-buzz [x]
  [x (zipmap fizz-buzz-specs
             (map #(s/conform % x) fizz-buzz-specs))])
\end{verbatim}
\begin{itemize}
\item Which describes parse result in a tuple
\begin{verbatim}
;; e.g. (spec-parse-buzz 1)
[1 #:conformer.buzz{:fizzes :clojure.spec.alpha/invalid,
                    :buzzes :clojure.spec.alpha/invalid,
                    :number 1}]
\end{verbatim}
\end{itemize}
\end{itemize}
\end{frame}
\begin{frame}[label={sec:org0f15e6c},fragile]{decomplect context (parse, don't calculate or interpret)}
 \begin{itemize}
\item Accumulate parser results like this
\begin{verbatim}
(into {} (map spec-parse-buzz [3 15 "lol"]))
\end{verbatim}
\begin{itemize}
\item A hash-map with number assoc'd with parse result
\begin{verbatim}
{3
 #:conformer.buzz{:fizzes 3,
                  :buzzes :clojure.spec.alpha/invalid,
                  :number 3},
 15
 #:conformer.buzz{:fizzes 15,
                  :buzzes 15,
                  :number 15},
 "lol"
 #:conformer.buzz{:fizzes :clojure.spec.alpha/invalid,
                  :buzzes :clojure.spec.alpha/invalid,
                  :number :clojure.spec.alpha/invalid}}
\end{verbatim}
\end{itemize}
\end{itemize}
\end{frame}
\begin{frame}[label={sec:orge9b0802}]{decomplect context (wicked pprint Buzz)}
\begin{verse}
\emph{"Let no number escape fizzbuzzness when showing itself."}\\
\vspace*{1em}
- \uline{\href{https://twitter.com/rdivyanshu}{@rdivyanshu}}\\
\vspace*{1em}
(Truly a genetlenerd and a scholar.)\\
\end{verse}
\end{frame}
\begin{frame}[label={sec:org164e993},fragile]{decomplect context (wicked pprint Buzz)}
 \begin{itemize}
\item Write a plain ol' function to pretty-print custom format
\begin{verbatim}
(ns pprint.buzz)
(require '[clojure.pprint :as pp])
(defn pprint-buzz [n]
  (let [divisible? (comp zero? rem)
        prettyprint
        (comp prn (partial format "%d doth %s"))]
    (cond
      (divisible? n 15) (prettyprint n "FizzBuzzeth")
      (divisible? n 3) (prettyprint n "Fizzeth")
      (divisible? n 5) (prettyprint n "Buzzeth")
      :else (prettyprint n
                         "not Fizzeth nor Buzzeth"))))
\end{verbatim}
\end{itemize}
\end{frame}
\begin{frame}[label={sec:org4038ae5},fragile]{decomplect context (wicked pprint Buzz)}
 \begin{itemize}
\item Hotpatch \href{https://github.com/clojure/clojure/blob/b1b88dd25373a86e41310a525a21b497799dbbf2/src/clj/clojure/pprint/dispatch.clj\#L175}{pprint dispatcher} to
use \texttt{pprint-buzz} formatter for all Numbers!
\begin{verbatim}
(#'pp/use-method pp/simple-dispatch
                 java.lang.Number
                 pprint-buzz)
\end{verbatim}
\item Enjoy a nondestructive, hilarious FizzBuzz experience!
\begin{verbatim}
(doseq [n [1 3 5 15]] (pp/pprint n)) ;; see REPL :)
\end{verbatim}
\item This is a joke implementation of a serious idea\ldots{}
pretty-printing data is open, fully extensible, and
can be done (and undone) in a live runtime.
\end{itemize}
\end{frame}
\section{Suddenly, Transducers}
\label{sec:orgc14b7fb}
\begin{frame}[label={sec:org3b5440c}]{decomplect what, now? (Suddenly, Transducers.)}
You:
\begin{quote}
\emph{Uh, what more could we possibly decomplect?}
\end{quote}
Clojure:
\begin{quote}
(whatever, input -> whatever) -> (whatever, input -> whatever)
\end{quote}
Rich Hickey:
\begin{quote}
\uline{\emph{\href{https://www.youtube.com/watch?v=6mTbuzafcII\&t=1677s}{Seems like a good project for the bar, later on.}}}
\end{quote}
\end{frame}
\begin{frame}[label={sec:org9842793}]{decomplect what, now? (Suddenly, Transducers.)}
\begin{itemize}
\item Data source
\begin{itemize}
\item sequence, stream, channel, socket etc.
\end{itemize}
\item Data sink
\begin{itemize}
\item sequence, stream, channel, socket etc.
\end{itemize}
\item Data transformer
\begin{itemize}
\item function of any value -> any other value
\end{itemize}
\item Data transformation process
\begin{itemize}
\item mapping, filtering, reducing etc.
\end{itemize}
\item Some process control
\begin{itemize}
\item Transduce finite data (of course)
\item Transduce streams
\item With optional early termination in either case
\end{itemize}
\end{itemize}
\end{frame}
\begin{frame}[label={sec:orgcca840f},fragile]{decomplect whatever (some setup)}
 \begin{itemize}
\item Bring back the usual suspects (this is key\ldots{} reuse logic)
\begin{verbatim}
(ns transducery.buzz)

(def divisible? (comp zero? rem))

(defn basic-buzz [n]
  (cond
    (divisible? n 15) "FizzBuzz"
    (divisible? n 3) "Fizz"
    (divisible? n 5) "Buzz"
    :else n))
\end{verbatim}
\end{itemize}
\end{frame}
\begin{frame}[label={sec:orgcbf7b8e},fragile]{decomplect Computation and \emph{Output} format (Demo 1)}
 \begin{itemize}
\item Separately define \emph{only} the transformation "xform"
\begin{verbatim}
(def fizz-buzz-xform
  (comp (map basic-buzz)
        (take 100))) ;; early termination
\end{verbatim}
\item Separately define \emph{only} input data
\begin{verbatim}
(def natural-nums (rest (range)))
\end{verbatim}
\end{itemize}
\end{frame}
\begin{frame}[label={sec:org41956be},fragile]{decomplect Computation and \emph{Output} format (Demo 1)}
 \begin{itemize}
\item Compose in various ways
\begin{itemize}
\item To produce a sequence
\begin{verbatim}
(transduce fizz-buzz-xform ;; calculator step
           conj ;; output method
           []   ;; output sink
           natural-nums) ;; input source
\end{verbatim}
\item To produce a string
\begin{verbatim}
(transduce fizz-buzz-xform
           str
           ""
           natural-nums)
\end{verbatim}
\item To produce a CSV string
\begin{verbatim}
(transduce (comp fizz-buzz-xform
                 (map #(str s "," %)))
           str
           ""
           natural-nums)
\end{verbatim}
\end{itemize}
\end{itemize}
\end{frame}
\begin{frame}[label={sec:org2cc110a},fragile]{decomplect Computation and \emph{Output} format (Demo 1)}
 \begin{itemize}
\item Consider:
\begin{itemize}
\item Parts \emph{not} modified, though \emph{output} and/or \emph{xform} modded
\item Effort needed to reuse fizz-buzzes other than \texttt{basic-buzz}?
\end{itemize}
\item Try it!
\end{itemize}
\end{frame}
\begin{frame}[label={sec:orgd89be41},fragile]{decomplect Computation and \emph{Input} format (Demo 2)}
 \begin{itemize}
\item Setup
\begin{verbatim}
(ns transducery.buzz)
(def numbers-file
  "Plaintext file containing numbers in some format."
  "/tmp/deleteme-spat-by-clj-fizz-buzz-demo.txt")
#_(spit numbers-file
        (clojure.string/join "\n" (range 1 10001)))
#_(slurp numbers-file)
\end{verbatim}
\end{itemize}
\end{frame}
\begin{frame}[label={sec:org2a66945},fragile]{decomplect Computation and \emph{Input} format (Demo 2)}
 \begin{itemize}
\item Transduce! Now, over file data.
\begin{verbatim}
(transduce (comp (map #(Integer/parseInt %))
                 fizz-buzz-xform) ;; calculator step
           conj ;; output method
           []   ;; output sink
           (clojure.string/split-lines
            (slurp numbers-file))) ;; input source
\end{verbatim}
\item Split-lines and file slurpin' still complected!!!
\begin{itemize}
\item decomplect clues in Tim Baldridge's (excellent) \uline{\href{https://tbaldridge.pivotshare.com/categories/transducers/2426/media}{tutorials}}
\end{itemize}
\end{itemize}
\end{frame}
\begin{frame}[label={sec:org5aad46d},fragile]{decomplected xform as a standalone calculator (Demo 3)}
 \begin{itemize}
\item The xform can still calculate just a single item
\begin{verbatim}
(ns transducery.buzz)
((fizz-buzz-xform conj) [] 3) ;; => ["Fizz"]
((fizz-buzz-xform str) "" 3) ;; => "Fizz"
((fizz-buzz-xform str) "" 1) ;; => "1"
((fizz-buzz-xform (fn [_ out] out)) nil 3) ;; "Fizz"
((fizz-buzz-xform (fn [_ out] out)) nil 1) ;; 1
\end{verbatim}
\item Meditate:
\begin{itemize}
\item The transducer's mandate of \emph{a la carte} re-composition \emph{demands}
that \emph{all} the new pulling apart \emph{must be fully compatible}
with \emph{all} the old pulling apart.
\end{itemize}
\end{itemize}
\end{frame}
\section{Fin}
\label{sec:orga1c7f74}
\begin{frame}[label={sec:orgbdcbc83},fragile]{So Long And Thanks For All The \(\lambda\)s}
 \begin{itemize}
\item Acknowledgements
\begin{verse}
To everyone who reviewed drafts, gave\\
feedback, ideas, encouragement, time\ldots{}\\
\ldots{} friends, fellow Clojurian Slack-ers,\\
sundry gentlenerds, one's better half,\\
and you dear reader.\\
\vspace*{1em}
May the Source be with you.\\
\vspace*{1em}
\(\lambda\) \texttt{<3}\\
\end{verse}
\item Blog post: \uline{\href{https://www.evalapply.org/posts/n-ways-to-fizzbuzz-in-clojure/}{evalapply.org/posts/n-ways-to-fizzbuzz-in-clojure}}
\item Email complaints or fizzbuzzes to
\begin{itemize}
\item fizzbuzz@evalapply.org
\end{itemize}
\end{itemize}
\end{frame}
\begin{frame}[label={sec:org5b1e96f}]{Buzz}
Ideas on deck, to put self on the hook\ldots{}

\begin{itemize}
\item[{$\square$}] curried fizzbuzz (like Ring libraries),

\item[{$\square$}] concurrent fizzbuzz (with agents)

\item[{$\square$}] advanced transducing fizzbuzz
(xform all the fizz-buzzes in all ways)

\item[{$\square$}] Maaaybe re-do Rich's ants sim
4 species of, ah, ConcurrAnts:
FizzAnt, BuzzAnt, FizzBuzzAnt, IdentiAnt

\item[{$\square$}] Outside of clojure.core?
maaybe core.async if not too contrived
\end{itemize}
\end{frame}
\end{document}
